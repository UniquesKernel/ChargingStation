\section{Charging Station Design}

Our design is based on the design suggested in the suplied material, and relies heavily on interfaces to allow testing the system through hardware simulations. It is thus easy to create a piece of code that sends the same signals to the system as the physical hardware implementation would, allowing for testing of the individual units before implementation. This is taken further, as the internal components of the system ( \textit{stationControl} and \textit{ChargeController}) interacts through an interface, allowing the \textit{chargeControlle} to be replaced for upgrades or testing of \textit{stationControl}.

\vspace{3em}

The design of the charging station is displayed in \autoref{fig:class-diagram}. Central to the system is the \textit{stationControl} class, which handles all interaction with the user, through various interfaces (display, door and rfid detector). It also interacts with the \textit{chargeController} through another interface. The \textit{chargeController} meanwhile interfaces with the USB charger, controlling it, and also displays charging status on the display through the display interface. 

\begin{figure}[h]
  \centering
  \includegraphics[width=\textwidth]{02-Body/images/ChargingStation_classDiagram.pdf}
  \caption{Class diagram for the charging station system}
  \label{fig:class-diagram}
\end{figure}

The charging station is first activated by the door opening, causing the station to ask the user to connect the phone. The user then connects the phone and closes the door, which causes the station to ask for an RFID. If the phone has not been correctly connected (as cheked by the chargeController) the station reports this to the user through the display. If the phone has been properly connected when the RFID is detected the door locks and the station starts charging the phone. This continues until the charging is complete or the same RFID as was used to start the charging is detected. When the RFID of the user is detected after charging has begun, the system stops the charging and unlocks the door, allowing the user to disconnect their phone.
This sequence is illustrated in \autoref{fig:seq-diagram}.

\newpage

\begin{figure}[h]
  \centering
  \includegraphics[scale = .6, trim={0cm 2cm 0cm 0cm}]{02-Body/images/SEQpdf.pdf}
  \caption{Sequence diagram describing normal opperation of the charging station}
  \label{fig:seq-diagram}
\end{figure}

\newpage

\subsection{Test Design}
To test the charge controller we used BVA and EP to test the boundary testcases above, on and below the maximum and minimum current values, as well as testing the connection threshold value. 
We used the EP principle to equate the allowed values of the charger current.
To test the event of RFID we used the ZOMBIE Many principle to simulate two correct RFID and one correct and one wrong RFID tag events. For the door event we did not use any additional test principle as the door operations and lock and unlock functions are expected to operate mechanically and fault in this systems will introduce testcases that are not a scope of this assignment. Display simulator test is a simple whitebox test, any attempt to test the consolewriteline method will move the focus of the assignment to begin testing Windows System Calls which we do not want to. This will reduce the coverage of the display simulator class but is acceptable.