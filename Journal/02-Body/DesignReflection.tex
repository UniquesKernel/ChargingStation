\section{Design Reflection}

The major component in design philosophy for the project have been to 
allow for a low coupled and testable design. In extension of that
the use of events in the observer pattern means that it has 
been unnessesary to use threads to handle multiple events.

\subsection{SOLID}

Take cognizance of the SOLID principles. We have divvied up the respondsible 
of each part of the ChargingStation system into seperate classes. 

This seperation of responsibility between the class makes it effortless to change 
one part of the system without impacting or breaking the other functionalities.
It also ensures that each class can be tested seperately. 

All components of the system is assembled with the use of Interfaces to ophold the
open and close principles. It makes it painless to extend the functionalities of
the system without having to change existing implementaions.

The principles of Interface segregation and dependency inversion are used extensively
to ensure that Station controller (a high level module) in no way depends on specific
implementations of its components but rather interface abstractions. 


\subsection{Events}

To handle event, we implemented the observer pattern to handle any changes in values
either from user actions or through time based events such a regular variable updates
with regular time intervals.

While such things could have been handled with the use of threads, we chose to use
event driven programming instead as implementing threaded system can be more complex
and harder to test.

\subsection{Conclusion}

While the SOLID principles makes everything loosely coupled, it does mean a marginally more
complex codebase as the solution increases in scope. It is however, the better solution
for the reasons mentioned above, making it lossely coupled and easilier testable.

The use of event driven programming rather than threaded programming follow the same 
argument where testability and simplicity was chosen of complexity.

In conclusion, due to these design considerations it means that mock classes could be 
used to ensure a high test coverage where each aspect of a class is regiously tested.

\newpage
